
\documentclass[12pt]{article}
\usepackage{epsfig,amsfonts,amssymb,amsmath,cite,epic}

%\setlength{\topmargin}{.5cm}
%\setlength{\textheight}{23.1cm}
\setlength{\textwidth}{14.5cm}
%\setlength{\evensidemargin}{2.4cm}
%\setlength{\oddsidemargin}{2.4cm} 
% \renewcommand{\baselinestretch}{1.1}
% 
    \def\CI {{\cal I}}
    \def\CF {{\cal F}}

\usepackage{marginnote}
 \reversemarginpar
 \newcommand{\newnote}{ \marginnote{\bf New in v2.0:}[-4mm]}
 

% from Bourjaily arXiv:1212.6974
%\pdfoutput=1
%\let\ifpdf\relax
\usepackage{color}
\let\normalcolor\relax

\newcommand{\mathematica}[3]{\vspace{0.35cm}\noindent\boxed{\begin{minipage}{#1\textwidth}\begin{tabular}{lp{13cm}}{\color{paper_blue}{\scriptsize{\tt In[1]:}}\raisebox{-0.65pt}{{\scriptsize{\tt=}}}}&{\tt #2}\\{\color{paper_blue}{\scriptsize {\tt Out[1]:}}\raisebox{-0.65pt}{{\scriptsize{\tt=}}}}&{\tt #3}\end{tabular}\end{minipage}}\vspace{0.35cm}}

\definecolor{varcolor}{rgb}{0.1,0.55,0.25}
\definecolor{functioncolor}{rgb}{0.1,0.35,0.75}
\definecolor{paper_blue}{rgb}{0.3,0.2,0.75}
\definecolor{paper_red}{rgb}{0.65,0.1,0.15}
\definecolor{paper_green}{rgb}{0.05,0.35,0.125}
\definecolor{paper_grey}{gray}{0.375}
\definecolor{perm}{rgb}{0.1,0.45,0.85}
\definecolor{deemph}{rgb}{0.7,0.7,0.7}
\setlength{\leftmargini}{21pt}
\newcommand{\vardef}[1]{{\color{varcolor}{\sl #1}\rule[-1.05pt]{7.5pt}{.75pt}}}
\newcommand{\vardefms}[1]{{\color{varcolor}{\sl #1}\rule[-1.05pt]{15pt}{.75pt}}}
\newcommand{\vardefo}[1]{{\color{varcolor}{\sl #1}\rule[-1.05pt]{7.5pt}{.75pt}{\bf{\sl :}}}}
\newcommand{\vardefoo}[1]{{\color{varcolor}{\sl #1}\rule[-1.05pt]{15pt}{.75pt}{\bf{\sl :}}}}
\newcommand{\defn}[3]{~\\[-35pt]\begin{itemize}\item[]\indent\hspace{-21pt}$\bullet$\hspace{-.75pt} {\tt {\color{functioncolor}#1}\![}#2{\tt\,]\!:}#3\end{itemize}\vspace{-7pt}
}
\newcommand{\defnNA}[3]{~\\[-30pt]\begin{itemize}\item[]\indent\hspace{-21pt}$\bullet$\hspace{-.75pt} {\tt {\color{functioncolor}#1}\!}#2{\tt\,\!:}#3\end{itemize}\vspace{-10pt}}
\newcommand{\defntb}[4]{~\\[-30pt]\begin{itemize}\item[]\indent\hspace{-21pt}$\bullet$\hspace{-.75pt} {\tt {\color{functioncolor}#1}\![}#2{\tt\,]\![}#3{\tt\,]\!:}#4\end{itemize}\vspace{-10pt}}
\newcommand{\defvar}[2]{~\\[-30pt]\begin{itemize}\item[]\indent\hspace{-21pt}$\bullet$\hspace{-.75pt} \var{#1}: #2\end{itemize}\vspace{-10pt}}
\newcommand{\var}[1]{{\tt{\color{varcolor}{\sl#1}}}}
\newcommand{\ind}{\hspace{4ex}}
\newcommand{\fun}[1]{{\color{functioncolor}#1}}

%%%%%%%%%%%%%%%

\newcommand{\hab}{}



\newcommand{\bZ}{\mathbb{Z}}
\newcommand{\cC}{\mathcal{C}}
\newcommand{\cQ}{\mathcal{Q}}

\newcommand{\IZ}{\mathbb{Z}}
\newcommand{\kk}{k}

\newcommand{\IR}{\mathbb{R}}


\def\ZZZ{{\hbox{ Z\kern-1.6mm Z}}}
\def\RRR{{\hbox{ R\kern-2.4mm R}}}
\def\CCC{{\hbox{ C\kern-2.0mm C}}}
\def\zzz{{\hbox{z\kern-1mm z}}}

\newcommand{\ten}{{(10)}}
\newcommand{\bet}{{( b )}}

\newcommand{\qq}{k}
\newcommand{\pp}{l}
\newcommand{\nn}{\nonumber \\}

\newcommand{\vt}{\vartheta}

\newcommand{\vtau} {\vec \tau}
\newcommand{\vj} {\vec J}
\newcommand{\vxi} {\vec \xi}
\newcommand{\vu} {\vec u}
\newcommand{\htau} {\vec \eta}
\newcommand{\vc}{\vec\chi}
\newcommand{\vpsi} {\vec \psi}

\newcommand{\qeq}{{\hbox{=\kern-2.3mm ? \kern.5mm }}}
\renewcommand{\qeq}{=}



\newcommand{\rrho}{r}
\newcommand{\bA}{{\bf A}}
\newcommand{\tx}{\wt x}
\newcommand{\bG}{{\bf G}}
\newcommand{\bF}{{\bar F}}
\newcommand{\bbb}{{\bar b}}
\newcommand{\gam}{\tau}
\newcommand{\eps}{\epsilon}
\newcommand{\vareps}{\varepsilon}
\newcommand{\ra}{\rangle}
\newcommand{\la}{\langle}
\newcommand{\T}{\chI_{T}(k)}
\newcommand{\Tm}{\chI_{T}(k')}
\newcommand{\Cn}{{\cal C}_n}
\newcommand{\vp}{\varphi}
\newcommand{\ve}{\varepsilon}
\newcommand{\tl}{\lambda}
\newcommand{\dt}{(\vec \nabla T)^2}
\newcommand{\hp}{{\wh\Phi}}
\newcommand{\hq}{{\wh Q_B}}
\newcommand{\he}{{\wh\eta_0}}
\newcommand{\ha}{{\wh{A}}}
\newcommand{\lllb}{\Bigl\langle\Bigl\langle}
\newcommand{\rrrb}{\Bigr\rangle\Bigr\rangle}
\newcommand{\tf}{\wt f}
\newcommand{\sss}{{\cal L}_{av}}
\newcommand{\bx}{\bar x}
\newcommand{\bw}{\bar w}
\newcommand{\ws}{{\wt\sigma}}
\newcommand{\wrh}{{\wt\rho}}
\newcommand{\wv}{{\wt v}}





\newcommand{\vv} {\bar v}
\newcommand{\uu} {\bar u}
\newcommand{\K}{{\rm K_1}}
\newcommand{\Kt}{{\rm \widetilde K_1}}

\newcommand{\B}{b'}
\newcommand{\C}{c\,'}
\newcommand{\bB}{\bar b'}
\newcommand{\Bu}{B_{\vec u}}
\newcommand{\VV}{{\cal V}}
\newcommand{\BB}{{\cal B}}
\newcommand{\DD}{{\cal D}}
\newcommand{\BBB}{{\cal B}}
\newcommand{\II}{{\cal I}}
\newcommand{\AAA}{{\cal A}}
\newcommand{\GG}{{\cal G}}
\newcommand{\KK}{{\cal K}}
\newcommand{\fff}{{\bf f}}
\newcommand{\ccc}{{\bf c}}
\newcommand{\FF}{{\cal F}}
\newcommand{\JJ}{{\cal J}}
\newcommand{\HH}{{\cal H}}
\newcommand{\MM}{{\cal M}}
\newcommand{\CC}{{\cal C}}
\newcommand{\bC}{{\bf C}}
\newcommand{\OO}{{\cal O}}
\newcommand{\QQ}{{\cal Q}}
\newcommand{\PP}{{\cal P}}
\newcommand{\EE}{{\cal E}}
\newcommand{\LL}{{\cal L}}
 \newcommand{\rrr}{\rangle\rangle}
\newcommand{\half}{{1\over 2}}
\newcommand{\wt}{\widetilde}
\newcommand{\wh}{\widehat}
\newcommand{\wc}{\wt}
\newcommand{\wb}{\bar}
%\newcommand{\bd}{\bar{\rm D}}
\newcommand{\RR}{{\cal R}}
\newcommand{\NN}{{\cal N}}
\newcommand{\TT}{{\cal T}}
\newcommand{\bg}{\bar g}
\newcommand{\ba}{\bar a}
\newcommand{\bc}{\bar c}
\newcommand{\bd}{\bar d}
\newcommand{\bb}{\bar b}
\newcommand{\bT}{\bar \Theta}
\newcommand{\SSS}{{\cal S}}
\newcommand{\tlx}{\left(\tilde \lambda ; X^0(0) \right)}
\newcommand{\al}{\alpha}

\newcommand{\tk}{\tilde \kappa}

%\newcommand{\gcd}{{\rm gcd}}
\newcommand{\ppp}{\prime\prime}

\newcommand{\omk}{\omega_n(\vec k)}
\newcommand{\onk}{\omega^{(N)}_{\vec k_\perp}}
\newcommand{\tI}{\wt\II}
\newcommand{\hI}{\wh\II}
\newcommand{\nI}{\II}

\newcommand{\cp}{\check\Phi}
\newcommand{\cps}{\Psi}
\newcommand{\crh}{\check\rho}
\newcommand{\cs}{\check\sigma}
\newcommand{\cv}{\check v}
\newcommand{\com}{\check\Omega}

\newcommand{\be}{\begin{equation}}
\newcommand{\ee}{\end{equation}}
\newcommand{\ben}{\begin{eqnarray}\displaystyle}
\newcommand{\een}{\end{eqnarray}}

\newcommand{\refb}[1]{(\ref{#1})}
\newcommand{\p}{\partial}

\newcommand\bOm{\bar\Omega}
\newcommand\tOm{\widetilde\Omega}

\newcommand{\zet}{\zeta}


\newcommand{\Lamb}{\Lambda}

%\renewcommand{\vec}{}

\def\one{{\hbox{ 1\kern-.8mm l}}}
\def\zero{{\hbox{ 0\kern-1.5mm 0}}}

\def\wa{{\wh a}}
\def\wb{{\wh b}}
\def\wc{{\wh c}}
\def\wdd{{\wh d}}
\newcommand{\cN}{\mathcal{N}}
\newcommand{\cM}{\mathcal{M}}
\def\Tr{\,{\rm Tr}\, }


\newcommand{\ta}{\tilde\alpha}
\newcommand{\tc}{\tilde d}
\renewcommand{\ha}{\hat\alpha}
\newcommand{\ca}{\check \alpha}
\newcommand{\cc}{\tilde c}
\newcommand{\dx}{c}

\newcommand{\gref}{g_{\rm Coulomb}}
\newcommand{\gR}{G_{\rm Higgs}}
\newcommand{\gRa}{g_{\rm Higgs}}
\newcommand{\QR}{Q_{\rm Higgs}}
\newcommand{\bQR}{\bar Q_{\rm Higgs}}
\newcommand{\QC}{Q_{\rm Coulomb}}
\newcommand{\bQC}{{\bar Q}_{\rm Coulomb}}
\newcommand{\gC}{G_{\rm Coulomb}}
\newcommand{\OmS}{\Omega^{\rm S}} 

%\newcommand{\eqref}{\refb}

\renewcommand{\theequation}{\thesection.\arabic{equation}}

\newcommand{\href}[1]{#1}

\def\newdefy{\vardef{y}}

\title{CoulombHiggs.m v2.0}
 
\author{Jan Manschot, Boris Pioline, Ashoke Sen}

\begin{document}

\maketitle

\baselineskip 15pt

%\tableofcontents



The  {\sc Mathematica} package {\tt CoulombHiggs.m} allows to compute the Poincar�-Laurent polynomial of the moduli space of stable representations of quivers using the {\it Coulomb branch} and {\it Higgs branch} formulae. The latter is based on Reineke's solution to the Harder-Narasimhan recursion \cite{MR1974891} and applies to quivers without oriented closed loops, while the former is based on a physical picture of BPS states as bound states of elementary 'single-centered' constitutents, and applies to any quivers with or without oriented loops \cite{1011.1258,1103.1887,Manschot:2012rx}.  This package was released together with the preprint \cite{Manschot:2013rx} where a general algorithm for computing the index of the
quantum mechanics of multi-centered BPS black holes (the Coulomb index) was outlined.
The new version 2.0, released along with the preprint \cite{Manschot:2013}, allows to compute the
Dolbeault-Laurent polynomial, relax assumptions on single-centered indices for basis vectors, study the effect of generalized mutations, and more. 
The package file {\tt CoulombHiggs.m} and various example files 
can be obtained from the second named author's webpage,
\vskip 2mm

\centerline{\tt http://www.lpthe.jussieu.fr/\~\,pioline/computing.html}

\section{Basic usage}

Assuming that the file {\tt CoulombHiggs.m} is present in the user's {\sc Mathematica} Application 
directory, the package is loaded by entering 

\mathematica{1.0}{ <<CoulombHiggs`}{CoulombHiggs v 2.0 - A package for evaluating quiver invariants using the Coulomb and Higgs branch formulae. }

If the file  {\tt CoulombHiggs.m} has not yet been copied in the user's {\sc Mathematica} Application 
directory but is in the same directory as the notebook, evaluate instead

\mathematica{1.0}{SetDirectory[NotebookDirectory[]]; <<CoulombHiggs`}{CoulombHiggs v 2.0 - A package for evaluating quiver invariants using the Coulomb and Higgs branch formulae.}


The first main routine  is {\tt \color{functioncolor} CoulombBranchFormula}, whose basic usage is illustrated below: \footnote{Note the following changes in v2.0: the fugacity $y$ 
is no longer a parameter of \fun{CoulombBranchFormula} and \fun{QuiverBranchFormula},
and the former computes the Dolbeault polynomial in terms of $\OmS(\alpha_i,t)$, 
rather than expressing  the Poincar\'e polynomial in terms of $\OmS(\alpha_i)$.
Other changes are highlighted by margin notes below. }


\mathematica{0.9}{Simplify[CoulombBranchFormula[4\{\{0, 1, -1\},\{-1, 0, 
   1\}, \{1, -1, 0\}\}, \{1/2, 1/6, -2/3\}, \{1, 1, 1\}]]  
     }
   {$ 2+ \frac{1}{y^2}+y^2 +  \text{OmS}(\{1,1,1\},y,t) $
}


%\mathematica{0.9}{Simplify[CoulombBranchFormulaNew[4\{\{0, 1, -1\},\{-1, 0, 
%   1\}, \{1, -1, 0\}\}, \{1/2, 1/6, -2/3\}, \{1, 1, 1\}]]
%%   /.  \{\text{OmS[gam\underline{~}, y\smallwedge n\underline{~}] -> OmS[gam, t\smallwedge
%%   n]},  \text{OmS[gam\underline{~}, y] -> OmS[gam, t]}\}
%  }
%   {$ 2+ \frac{1}{y^2}+y^2 +  \text{OmS}(\{1,1,1\}, t) $
%}

This routine computes the Dolbeault-Laurent polynomial  of the quiver moduli space, expressed in terms of the single-centered indices. 
The first argument corresponds to the matrix of DSZ products $\alpha_{ij}$ (an antisymmetric matrix of integers), the second to the FI parameters $\zeta_i$ (a vector of rational numbers), the third to the dimension vector $N_i$ (a vector of integers). 
The variables
 $y$ and $t$ are fugacities conjugate to the sum of the Dolbeault
degrees $p+q$ (i.e. the angular
momentum) and to the difference of the Dolbeault
degrees $p-q$, respectively. The Poincar\'e-Laurent
polynomial is obtained by setting $t=1$. For generic superpotential, the single-centered
indices $\OmS(\gamma,y)\equiv \OmS(\gamma,y,1)$ are conjectured to be
independent of $y$.  In the above example,  the 
Dolbeault polynomial of the moduli space of a three-node Abelian cyclic quiver with $4$ arrows between each subsequent node is expressed in terms of the single-centered index
$\OmS(\gamma_1+\gamma_2+\gamma_3,y,t)$. 
The second main routine is {\tt \color{functioncolor} HiggsBranchFormula}, which computes the
Poincar\'e-Laurent polynomial  using the Higgs branch formula  (which is only valid for quivers without oriented  loop, but the routine works irrespective of this
assumption). The arguments are the same as for {\tt \color{functioncolor} CoulombBranchFormula}: 

\mathematica{1.0}{Simplify[HiggsBranchFormula[\{\{0, 3\},\{-3, 0\}\}, \{1/2,-1/2\}, \{2, 2\}]]
  }
   {$ -\frac{\left(y^2+1\right) \left(y^8+y^4+1\right)}{y^5} $
}

The above command computes the Poincar\'e-Laurent polynomial for the Kronecker quiver with 3 arrows, FI parameters $(1/2,-1/2)$, dimension vector $(2,2)$. The package allows for much
more, as documented below. 
Inline documentation can be 
obtained by typing e.g. 

\mathematica{1.0}{?CoulombBranchFormula}
   {}

\section{Symbols}

 
\defvar{y}{fugacity conjugate to the sum of Dolbeault degrees $p+q$ (i.e. angular momentum);}

\defvar{t}{fugacity conjugate to the difference of Dolbeault degrees $p-q$;}

\defn{Om}{\vardef{charge vector},\newdefy}{denotes the refined index  
$\Omega(\gamma,y)$;}

\defn{Omb}{\vardef{charge vector},\newdefy }{denotes the rational refined index
$\bar\Omega(\gamma,y)$;}


\defn{OmS}{\vardef{charge vector},\newdefy,\vardef{t} }{denotes the single-centered index 
$\OmS(\gamma,y,t)$. } \newnote %bp

\defn{OmS}{\vardef{charge vector},\newdefy}{denotes 
$\OmS(\gamma,y)\equiv \OmS(\gamma,y,t=1)$.}

\defn{OmS}{\vardef{charge vector}}{denotes 
$\OmS(\gamma,y)$, under the assumption that it is independent of $y$ (which is
conjectured to be the case for  generic superpotential)}


\defn{OmT}{\vardef{charge vector},\newdefy }{denotes the (unevaluated)  
function $\Omega_{\rm tot}(\gamma,y)$;}

\defn{Coulombg}{\vardef{list of charge vectors},\newdefy }{: denotes the (unevaluated) Coulomb index $\gref(\{\alpha_i\},\{c_i\},y)$, leaving the FI parameters unspecified;} 

\defn{HiggsG}{\vardef{charge vector},\newdefy }{denotes the (unevaluated)  
stack invariant $\gR(\gamma,y)$;}



\defn{CoulombH}{\vardef{list of charge vectors},\vardef{multiplicity vector},\newdefy }{denotes the (unevaluated) factor $H(\{\alpha_i\},\{n_i\},y)$ appearing in the formula  for 
$\Omega_{\rm tot}(\sum n_i\alpha_i,y)$  in terms of 
$\OmS(\alpha_i,y)$.}

\defn{QFact}{\vardef{n},\newdefy }{represents the (non-evaluated) $q$-deformed factorial $[n,y]!$}


\section{Environment variables}


\defvar{{\tt \$QuiverPerturb1}}{Sets the size of the perturbation $\epsilon_1=1/\var{\$QuiverPerturb}$ of the DSZ products, set to 1000 by default.}

\defvar{{\tt \$QuiverPerturb2}}{Sets the size of the perturbation $\epsilon_2=1/\var{\$DSZPerturb}$ of the DSZ products, set to $10^{10}$ by default.}


\defvar{{\tt \$QuiverNoLoop}}{If set to True, the quiver will be assumed to have no oriented loop, hence all $H$ factors and all $\OmS(\alpha)$ will be set to zero (unless $\alpha$ is a basis vector). Set to False by default.}

\defvar{{\tt \$QuiverTestLoop}}{If set to True, all $H$ factors and $\OmS(\alpha)$ corresponding to subquivers without loops will be set to zero (unless $\alpha$ is a basis vector).
Set to True by default.
Determining whether a subquiver has loops is time-consuming, so for large quivers it may be advisable to disable this feature. Note that 
\var{{\tt \$QuiverNoLoop}} takes precedence over this variable.}

\defvar{{\tt \$QuiverMultiplier}}{Overall scaling factor of the DSZ matrix in any evaluation of 
\var{\tt Coulombg} or \var{\tt HiggsG}. Set to 1 by default, could be a formal variable.}

\defvar{{\tt \$QuiverVerbose}}{If set to False, all consistency tests on data and corresponding error messages will be skipped. Set to True by default.}

\defvar{{\tt \$QuiverDisplayCoulombH}}{If set to True, the routine 
\fun{\tt CoulombBranchFormula} will return a list   $\{ \var{Q}, \var{R} \}$ where
 $\var{Q}$  is the Poincar\'e-Laurent   polynomial and \var{R} is a  list of replacement rules for the \var{CoulombH} factors. Set to False by default.}

\defvar{{\tt \$QuiverPrecision}}{Sets the numerical precision with which all consistency tests
are carried out. This is set to 0 by default since all data are assumed to be rational numbers. This can be set to a small real number when using real data, however the user is warned that rounding errors
tend to grow quickly.}



\defvar{{\tt \$QuiverRecursion}}{If set to 1 (default value), then the new recursion relations 
from \cite[v2]{Manschot:2013rx} are used for computing
\fun{\tt CoulombF};  if set to 0  the recursion relation
from \cite[v1]{Manschot:2013rx}  is used instead. } \newnote %bp

\defvar{{\tt \$QuiverOmSbasis}}{Set to 1 by default. If set to 0,  the routines 
\fun{\tt SimplifyOmSbasis} and \fun{\tt SimplifyOmSbasismult} are deactivated, so that 
 the assumption that basis 
vectors carry $\OmS(\ell\gamma_i)=\delta_{\ell,1}$ is relaxed.}  %bp
%\end{itemize}

\section{Coulomb index}

\defn{CoulombF}{\vardef{Mat},\vardef{Cvec}}{returns the index of collinear solutions 
$F(\{\ta_1,\cdots \ta_n\}, \{\cc_1,\cdots \cc_n\})$ with 
                   DSZ products $\ta_{ij}=\var{Mat}[[i,j]]$, FI terms $\cc_i=\var{Cvec}[[i]]$ 
                   and trivial ordering.}

\defn{CoulombG}{\vardef{Mat}}{returns the index of scaling collinear solutions 
$G(\{\ha_1,\cdots \ha_n\})$ 
with  DSZ products $\ha_{ij}=\var{Mat}[[i,j]]$ and trivial ordering. The total angular momentum 
                   $\sum_{i<j} Mat[[i,j]]$ must vanish;}

\defn{CoulombIndex}{\vardef{Mat},\vardef{PMat},\vardef{Cvec},\newdefy }{evaluates the Coulomb index $\gref(\{\alpha_1,\cdots $ $\alpha_n\};$ $\{\dx_1,\cdots \dx_n\};y)$ 
                   with DSZ products $\alpha_{ij}=\var{Mat}[[i,j]]$, perturbed to \var{PMat}[[i,j]] so as to lift 
                   accidental degeneracies, possibly rescaled by an overall factor of 
                   \var{{\tt \$QuiverMultiplier}}, 
                   FI terms $\dx_i=\var{Cvec}[[i]]$, angular momentum fugacity \var{y};
                   }


\defn{CoulombFNum}{\vardef{Mat}}{computes numerically the index $F(\{\ta_1,\dots \ta_n\},\{\cc_1,\dots \cc_n\})$  with DSZ matrix $\ta_{ij}=\var{Mat}[[i,j]]$ and FI parameters $\cc_i=\var{Cvec}[[i]]$. For testing purposes only, works for up to 5 centers.}

\defn{CoulombGNum}{\vardef{Mat}}{computes numerically the scaling index $G(\ha_1,\dots \ha_n)$  with DSZ matrix $\ha_{ij}=\var{Mat}[[i,j]]$. For testing purposes only, works for up to 6 centers.}
                   

\defn{CoulombIndexNum}{\vardef{Mat},\vardef{PMat},\vardef{Cvec},\vardef{k},\newdefy }{returns the Coulomb index $\gref(\{\alpha_1,\cdots $ $\alpha_n\};$ $\{\dx_1,\cdots \dx_n\};y)$ 
                   with DSZ products $\alpha_{ij}=\var{Mat}[[i,j]]$,  possibly rescaled by an overall factor of 
                   \var{{\tt \$QuiverMultiplier}},  
                   FI terms $\dx_i=\var{Cvec}[[i]]$, angular momentum fugacity \var{y}, by searching
                   collinear solutions numerically;    For testing purposes only, works for up to 5 centers.               }



\section{Coulomb branch formula}

 \defn{CoulombBranchFormula}{\vardef{Mat},\vardef{Cvec},\vardef{Nvec}}
	                                   {computes  the Dolbeault %Poincar\'e-Laurent
                   polynomial  of a quiver with DSZ products $\alpha_{ij}=\var{Mat}[[i,j]]$,
                   dimension vector $N_i=\var{Nvec}[[i]]$, 
                   FI parameters $\zeta_i=\var{Cvec}[[i]]$,
                   in terms of single-centered invariants $\OmS$.
                  This standalone routine first constructs the Poincar\'e-Laurent 
                  polynomial,
                   evaluates the Coulomb indices $\gref$, 
                   determines the $H$ factors recursively using the minimal modification hypothesis
                   and finally replaces $y$ by $t$ in the argument of $\OmS$ to construct
                   the Dolbeault polynomial. 
If \var{{\tt \$QuiverDisplayCoulombH}} is 
                   set to True, the routine  returns a list $\{ \var{Q}, \var{R} \}$,
                                    where $\var{Q}$ is the Poincar\'e polynomial and \var{R} is a  list of replacement rules for the \var{CoulombH} factors.
                                                       For quivers without loops, the process can be sped up greatly by setting \var{\tt \$QuiverNoLoop} to True.
                   For complicated quivers
                   it is advisable to implement the Coulomb branch formula step by step, using the 
                   more elementary routines described below. 
                   }


                   
 \defn{CoulombBranchFormulaFromH}{\vardef{Mat},\vardef{Cvec},\vardef{Nvec},\vardef{R} }
	                                   {returns the Dolbeault
                   polynomial  of a quiver with DSZ products $\alpha_{ij}=\var{Mat}[[i,j]]$,  
                   dimension vector $N_i=\var{Nvec}[[i]]$, 
FI parameters $\zeta_i=\var{Cvec}[[i]]$, using the rule 
           \var{R} to replace all \var{CoulombH} factors.}
                   


\defn{QuiverPoincarePolynomial}{\vardef{Nvec},\newdefy }{constructs the Poincar\'e-Laurent
                   polynomial of a quiver. Coincides with
                   \fun{\tt QuiverPoincarePolynomialRat}                   for primitive dimension vector;
                 }


\defn{QuiverPoincarePolynomialRat}{\vardef{Nvec},\newdefy }{constructs the rational Poincar\'e-Laurent
                   polynomial $\bQC(\gamma;\zeta;y)$;}
         
\defn{QuiverPoincarePolynomialExpand}{\vardef{Mat},\vardef{PMat},\vardef{Cvec},
\vardef{Nvec}, \vardef{Q}}{
                  evaluates the Cou-lomb 
                  indices $\gref$ and   total 
                   single-centered indices $\Omega_{\rm tot}(\alpha_i,y)$   appearing in the 
                  Poincar\'e-Laurent polynomial \var{Q} of a quiver with DSZ products
                  $\alpha_{ij}=\var{Mat}[[i,j]]$, perturbed to 
                   $\var{PMat}[[i,j]]$, 
                   dimension vector $N_i=\var{Nvec}[[i]]$, 
                 FI parameters $\zeta_i=\var{Cvec}[[i]]$;}


  

\defn{CoulombHSubQuivers}{\vardef{Mat},\vardef{PMat},\vardef{Nvec},\newdefy }{computes recursively all \var{CoulombH} factors for DSZ matrix \var{Mat}, perturbed to \var{PMat},
and any dimension vector strictly less than \var{Nvec}; relies on the next two routines:}

                  
                   \defn{ListCoulombH}{\vardef{Nvec},\vardef{Q}}{ returns
                   returns
                   a pair $\{\var{ListH},\var{ListC}\}$ where \var{ListH} is a list of \var{CoulombH}
                    factors possibly
                   appearing in the Poincar\'e-Laurent polynomial \var{Q }of a quiver with dimension vector 
                   \var{Nvec}, and \var{ListC} is the list of coefficients which multiply the monomials in 
                   $\OmS(\alpha_i,y)$ canonically associated to the $H$ factors in \var{Q}.}

         
\defn{SolveCoulombH}{\vardef{ListH},\vardef{ListC},
\vardef{R}}{ returns
                   a list of replacement rules for the \var{CoulombH} factors 
                   listed in \var{ListH}, by applying the minimal modification hypothesis
                   to the coefficients listed in \var{ListC}. The last argument  is 
                   a replacement rule for \var{CoulombH} factors associated to subquivers.}
                   

\defn{MinimalModif}{\vardef{f}}{returns the symmetric Laurent polynomial which coincides 
                   with the Laurent expansion expansion of the symmetric rational function $f$ at $y=0$, up to strictly positive powers of $y$. Here symmetric means invariant under $y\to 1/y$. In practice,
\fun{\tt  MinimalModif}[\var{f}] evaluates the contour integral in \cite{Manschot:2012rx}, Eq 2.9
\be
\label{uint}
\oint \frac{{\rm d} u}{2\pi {\rm i}} \frac{(1/u-u) \, f(u)}{(1-u y)(1-u/y)}
\ee 
by deforming the contour around 0 into a sum of counters over all poles of $f(u)$ and zeros
of $(1-uy)(1-u/y)$. This trick allows to compute \eqref{uint} even if the order of the pole of $f(y)$
at $y=0$ is unknown, which happens if  \var{{\tt \$QuiverMultiplier}} is a formal variable.
                   }

\defn{SimplifyOmSbasis}{\vardef{f}}{replaces  $\OmS(\gamma,y)\to 1$ when $\gamma$ is a basis vector, unless  \var{{\tt \$QuiverOmSbasis}} is set to 0; 
}  \newnote %bp

\defn{SimplifyOmSbasismult}{\vardef{f}}{replaces  $\OmS(\gamma,y)\to 0$ when $\gamma$   is a non-trivial multiple of a basis vector, unless  \var{{\tt \$QuiverOmSbasis}} is set to 0;} %bp



\defn{CoulombHNoLoopToZero}{\vardef{Mat},\vardef{f}}{sets to zero any $H$
                  factor in \var{f} corresponding to subquivers without loop, assuming DSZ products
                  $\alpha_{ij}=\var{Mat}[[i,j]]$
                  ; active only on 2-node subquivers if \var{{\tt \$QuiverTestLoop}} is set to False}


\defn{OmTNoLoopToZero}{\vardef{Mat},\vardef{f}}{sets to zero any $\Omega_{\rm tot}$
                  factor in \var{f} corresponding to subquivers without loop, assuming DSZ products
                  $\alpha_{ij}=\var{Mat}[[i,j]]$
                  ; active only on 2-node subquivers if \var{{\tt \$QuiverTestLoop}} is set to False
                  }

\defn{OmSNoLoopToZero}{\vardef{Mat},\vardef{f}}{sets to zero any $\OmS$  
                  factor in \var{f} corresponding to subquivers without loop, assuming DSZ products
                  $\alpha_{ij}=\var{Mat}[[i,j]]$
                  ; active only on 2-node subquivers if \var{{\tt \$QuiverTestLoop}} is set to False
                  }

                   \defn{EvalCoulombH3}{\vardef{Mat},\vardef{f}}{ evaluates any 3-center $H$ factor with multiplicity vector $\{1,1,1\}$ 
               appearing   in $f$. Not used in any routine so far.}

\defn{DropFugacity}{\vardef{f}}{replaces $\OmS(\gamma,y^m,t^m)$ by $\OmS(\gamma,t^m)$ everywhere in $f$} \newnote %bp
                
\defn{SwapFugacity}{\vardef{f}}{replaces $\OmS(\gamma,y^m)$ with $\OmS(\gamma,y^m,t^m)$ everywhere in $f$} 


\section{Higgs branch formula}

\defn{HiggsBranchFormula}{\vardef{Mat},\vardef{Cvec},\vardef{Nvec} }
	                                   {computes the Poincar\'e-Laurent
                   polynomial  of a quiver with DSZ products $\alpha_{ij}=\var{Mat}[[i,j]]$ (possibly              rescaled by \var{{\tt \$QuiverMultiplier}}), dimension vector $N_i=\var{Nvec}[[i]]$, 
                   FI parameters $\zeta_i=\var{Cvec}[[i]]$, using the Higgs branch formula. 
                   It is assumed, but not checked, that the quiver has no oriented 
                   loop;}


\defn{StackInvariant}{\vardef{Mat},\vardef{Cvec},\vardef{Nvec},\newdefy }{gives 
                 the stack 
                  invariant 
                  of a quiver with DSZ matrix $\alpha_{ij}=\var{Mat}[[i,j]]$, possibly rescaled 
                  by an overall factor of 
                   \var{{\tt \$QuiverMultiplier}}, 
                  FI parameters $\zeta_i=\var{Cvec}[[i]]$,  dimension vector $N_i=\var{Nvec}[[i]]$, 		using Reineke's formula; the answer is written in terms of unevaluated
		$q$-deformed factorials \var{QFact[n,y]};}
		
		\defn{AbelianStackInvariant}{\vardef{Mat},\vardef{Cvec},\newdefy }{gives 
                 the Abelian stack 
                  invariant \eqref{edefghiggs} 
                  of a quiver with DSZ matrix $\alpha_{ij}=\var{Mat}[[i,j]]$, possibly rescaled 
                  by an overall factor of 
                   \var{{\tt \$QuiverMultiplier}}, 
                  FI parameters $\zeta_i=\var{Cvec}[[i]]$, 		
                  using Reineke's formula; coincides with \fun{\tt StackInvariant} 
                  with \var{Nvec}$=\{1,\dots 1\}$ except that tests of marginal or threshold stability
                  are performed (unless   \var{\tt \$QuiverVerbose} is set to False);
                  }
                   
\defn{QDeformedFactorial}{\vardef{n},\newdefy }{gives the $q$-deformed factorial $[n,y]!$}


\defn{EvalQFact}{\vardef{f}}{evaluates any \var{QFact[n,y]} appearing in \var{f}}


\section{Utilities}

\defn{ListAllPartitions}{\vardef{charge vector}}{returns the list of unordered 
                   partitions $\{\alpha_i\}$ of the positive integer vector $\gamma$ as a sum of positive, non-zero integer vectors $\alpha_i$; 
                   }
                   
                   \defn{ListAllPartitionsMult}{\vardef{charge vector}}{returns the list of unordered 
                   partitions $\{\alpha_i,m_i\}$ of the positive integer vector $\gamma$ as a sum of positive, non-zero integer vectors $\alpha_i$ with multiplicity $m_i$; 
                   }

\defn{ListSubQuivers}{\vardef{Nvec}}{gives a list of all dimension vectors less or equal to \var{Nvec};}


\defn{SubDSZ}{\vardef{Mat},\vardef{Cvec},\vardef{Li}}{gives the
                    DSZ matrix of the subquiver made of vectors in list \var{Li};}

 \defn{SymmetryFactor}{\vardef{Li}}{gives the symmetry factor $1/  |{\rm Aut}(\{\alpha_1, \alpha_2,\cdots, \alpha_n\}|$ for the list of charge vectors \var{Li};}

\defn{OmTRat}{\vardef{Nvec},\newdefy }{ gives the rational total invariant 
$\bar\Omega_{\rm tot}(\gamma;y)$ in terms of $\Omega_{\rm tot}(\gamma;y)$.
Coincides with the latter if $\gamma$ is primitive.}

\defn{OmTToOmS}{\vardef{f}}{expands out any $\Omega_{\rm tot}(\gamma;y)$ in $f$
 into $H$ factors and $\OmS$'s;}
           
\defn{OmToOmb}{\vardef{f}}{expresses any $\Omega(\gamma;y)$ in $f$
 in terms of $\bar\Omega(\gamma;y)$'s;}

\defn{OmbToOm}{\vardef{f}}{expresses  any $\bar\Omega(\gamma;y)$ in $f$
 in terms of $\Omega(\gamma;y)$'s;}

                 
\defn{HiggsGToOmb}{\vardef{Nvec},\newdefy }{Returns the (unevaluated) HN invariant
$\gR(\gamma,y)$ in terms of the rational refined indices $\Omega(\gamma;y)$;}

\defn{OmbToHiggsG}{\vardef{Nvec},\newdefy }{Returns the (unevaluated) rational refined index $\Omega(\gamma;y)$ in terms of the (unevaluated) 
stack invariants $\gR(\gamma,y)$;}



\defn{RandomCvec}{\vardef{Nvec}}{generates a random set of FI parameters $\zeta_i$ between -1 and 1, such that $\sum \zeta_i\, \var{Nvec}[[i]]=0$;}

\defn{UnitStepWarn}{\vardef{x}}{gives 1 for $x>0$, 0 for $x<0$, and $1/2$ if $x=0$. Produces a warning in this latter case, irrespective of the value of \var{\tt \$QuiverVerbose}. If so, the user
is advised run the 
computation again with a different random perturbation. 
}

\defn{GrassmannianPoincare}{\vardef{k},\vardef{n},\newdefy}{computes the Poincar\'e
polynomial of the Grassmannian $G(k,n)$ via Eq. (6.22) in \cite{Manschot:2012rx}.} \newnote%bp

\defn{CyclicQuiverOmS}{\vardef{avec},\vardef{t}}{computes the single-centered index 
$\OmS(\gamma_1,\dots, \gamma_K)$ associated
to a cyclic Abelian quivers with DSZ matrix $\alpha_{i,j+1}=\var{avec}[[i,i+1]]$ via Eq
(4.29) in \cite{Manschot:2012rx}.} %bp

\defn{QuiverPlot}{\vardef{Mat}}{Displays the quiver with DSZ matrix \var{Mat}.} %bp

\subsection{Mutations} \newnote %bp

The following routines and environment variables were introduced in  {\tt CoulombHiggs.m} v1.1,
to allow investigation of mutations of generalized quivers \cite{Manschot:2013}:

\defn{MutateRight}{\vardef{Mat},\vardef{Cvec},\vardef{Nvec},\vardef{k}}{
Computes the  DSZ matrix, FI parameters and dimension vector of the quiver obtained
by applying a right-mutation with respect to the node $k$. If \var{k} is a list $\{k_i\}$,  then the
right mutations $k_i$ are applied successively, starting from the last entry in \var{k}.  No
consistency check on the FI parameters is performed.}

\defn{MutateLeft}{\vardef{Mat},\vardef{Cvec},\vardef{Nvec},\vardef{k}}{
Computes the  DSZ matrix, FI parameters and dimension vector of the quiver obtained
by applying a left-mutation with respect to the node $k$. If \var{k} is a list $\{k_i\}$,  then the
right mutations $k_i$ are applied successively, starting from the last entry in \var{k}. No
consistency check on the FI parameters is performed.} 

\defn{OmStoOmS2}{\vardef{f}}{replaces $\var{\tt OmS[gam,y,t]}$ by $\var{\tt OmS2[gam,y,t]}$
anywhere in $\var{f}$. This is useful for distinguishing the single-centered invariants of the mutated
quiver from those of the original one.}

\defn{MutateRightOmS}{\vardef{Mat},\vardef{k},\vardef{f}}{expresses the single-centered
invariants $\var{\tt OmS[gam,y,t]}$ of the original quiver with DSZ matrix \var{Mat} in terms of the single-centered
invariants $\var{\tt OmS2[gam,y,t]}$ of the quiver obtained by right-mutation with respect to node $k$,
using Eq. 1.13 in \cite{Manschot:2013}.}

\defn{MutateLeftOmS}{\vardef{Mat},\vardef{k},\vardef{f}}{expresses the single-centered
invariants $\var{\tt OmS[gam,y,t]}$ of the original quiver with DSZ matrix \var{Mat} in terms of the single-centered
invariants $\var{\tt OmS2[gam,y,t]}$ of the quiver obtained by left-mutation with respect to node $k$,
using Eq. 1.13 in \cite{Manschot:2013}.}

\defn{MutateRightOmS2}{\vardef{Mat},\vardef{k},\vardef{f}}{expresses the single-centered
invariants $\var{\tt OmS2[gam,y,t]}$ a quiver with DSZ matrix \var{Mat} in terms of the single-centered
invariants $\var{\tt OmS[gam,y,t]}$ of the quiver obtained by right-mutation with respect to node $k$.
Identical to \fun{\tt MutateRightOmS}, except for swapping $\var{\tt OmS[gam,y,t]}$ and $\var{\tt OmS2[gam,y,t]}$.}

\defn{MutateLefttOmS2}{\vardef{Mat},\vardef{k},\vardef{f}}{expresses the single-centered
invariants $\var{\tt OmS2[gam,y,t]}$ a quiver with DSZ matrix \var{Mat} in terms of the single-centered
invariants $\var{\tt OmS[gam,y,t]}$ of the quiver obtained by right-mutation with respect to node $k$.
Identical to \fun{\tt MutateLeftOmS}, except for swapping $\var{\tt OmS[gam,y,t]}$ and $\var{\tt OmS2[gam,y,t]}$.}

\defn{DropOmSNeg}{\vardef{f}}{equates to 0 any $\OmS(\gamma,y,t)$ where the dimension
vector associated to $\gamma$ has negative components.}

\defvar{{\tt \$QuiverMutationMult}}{Equal to 1 by default. Set to $M$, defined in Eq. (1.8) of
  \cite{Manschot:2013} when dealing with generalized quivers.}  


\begin{thebibliography}{9}

\bibitem{MR1974891}
M.~Reineke, ``The {H}arder-{N}arasimhan system in quantum groups and cohomology
  of quiver moduli,'' {\em Invent. Math.} {\bf 152} (2003), no.~2, 349--368.

\bibitem{1011.1258} 
  J.~Manschot, B.~Pioline and A.~Sen,
  ``Wall Crossing from Boltzmann Black Hole Halos,''
  JHEP {\bf 1107}, 059 (2011)
  [arXiv:1011.1258 [hep-th]].
  %%CITATION = ARXIV:1011.1258;%%

\bibitem{1103.1887} 
  J.~Manschot, B.~Pioline and A.~Sen,
  ``A Fixed point formula for the index of multi-centered N=2 black holes,''
  JHEP {\bf 1105}, 057 (2011)
  [arXiv:1103.1887 [hep-th]].
  %%CITATION = ARXIV:1103.1887;%%

%\cite{Manschot:2012rx}
\bibitem{Manschot:2012rx}
  J.~Manschot, B.~Pioline and A.~Sen,
  ``From Black Holes to Quivers,''
  JHEP {\bf 1211} (2012) 023
  [arXiv:1207.2230 [hep-th]].
  %%CITATION = ARXIV:1207.2230;%%

\bibitem{Manschot:2013rx}
  J.~Manschot, B.~Pioline and A.~Sen,
  ``On the Coulomb and Higgs branch formulae for multi-centered black holes and quiver invariants,''
  JHEP {\bf 1305} (2013) 166
  [arXiv:1302.5498 [hep-th]].
  %%CITATION = ARXIV:1302.5498;%%
  %3 citations counted in INSPIRE as of 26 Sep 2013
  
\bibitem{Manschot:2013}
 J.~Manschot, B.~Pioline and A.~Sen,
  ``Generalized quiver mutations and single-centered invariants'',
    [arXiv:1309.nnnn [hep-th]].

\end{thebibliography}

\end{document}